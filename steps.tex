Install packages required : 
sequelize
global package : mysql2 sequelize-cli

Creating models:
Post Model--
Command: sequelize model:generate --name Post --attributes title:string,content:text,imageUrl:string,categoryId:integer,userId:integer

Output:
Sequelize CLI [Node: 18.15.0, CLI: 6.6.2, ORM: 6.35.2]

New model was created at C:\Users\Mohd Abuzaid\Desktop\Website Projects\Node-MySQL-BlogApp\api\models\post.js .
New migration was created at C:\Users\Mohd Abuzaid\Desktop\Website Projects\Node-MySQL-BlogApp\api\migrations\20240119200740-create-post.js

User Model:
Command: sequelize model:generate --name User --attributes name:string,email:string,password:string

Output:
Sequelize CLI [Node: 18.15.0, CLI: 6.6.2, ORM: 6.35.2]

New model was created at C:\Users\Mohd Abuzaid\Desktop\Website Projects\Node-MySQL-BlogApp\api\models\user.js .
New migration was created at C:\Users\Mohd Abuzaid\Desktop\Website Projects\Node-MySQL-BlogApp\api\migrations\20240119201419-create-user.js .


Next : Start Xampp and start Apache and MySql Server
then head up to http://localhost/phpmyadmin/
Create Database there
Then setup config.json and start migrating of models to database to create tables.

Migration command:
sequelize db:migrate

Sequelize CLI [Node: 18.15.0, CLI: 6.6.2, ORM: 6.35.2]

Loaded configuration file "config\config.json".
Using environment "development".
== 20240119201228-create-post: migrating =======
== 20240119201228-create-post: migrated (0.019s)

== 20240119201419-create-user: migrating =======
== 20240119201419-create-user: migrated (0.014s)

== 20240119201541-create-category: migrating =======
== 20240119201541-create-category: migrated (0.013s)

== 20240119201730-create-comment: migrating =======
== 20240119201730-create-comment: migrated (0.013s)






Definition:
Sequelize is a promise-based Node.js ORM for Postgres, MySQL, MariaDB, SQLite, and Microsoft SQL Server. Its features are solid transaction support, relations, eager and lazy loading, read replication and many more.

Features of Sequelize:

Sequelize is a third-party package to be precise its an Object-Relational Mapping Library(ORM)..
Standardization ORMs usually have a single schema definition in the code. This makes it very clear what the schema is, and very simple to change it.
No need to learn SQL – queries are written in plain JavaScript.



`Sequelize CLI` (Command Line Interface) is a command-line tool provided by Sequelize to assist developers in managing database-related tasks, such as generating models, migrations, and seeds. It simplifies the process of interacting with Sequelize and helps automate common development tasks associated with database management.

Here are some key functionalities provided by Sequelize CLI:

1. **Initialization:** You can use Sequelize CLI to initialize a new Sequelize project by running the `sequelize init` command. This sets up the project structure, configuration files, and directories for models, migrations, and seeders.

   ```bash
   sequelize init
   ```

2. **Model Generation:** Sequelize CLI allows you to generate model files easily. Models define the structure of your data and how it is mapped to the database. The following command creates a new model:

   ```bash
   sequelize model:generate --name User --attributes username:string,email:string
   ```

   This command generates a model file for a `User` with `username` and `email` attributes.

3. **Migration Generation:** Migrations in Sequelize are used to manage changes to the database schema. Sequelize CLI can generate migration files based on changes you make to your models. For example:

   ```bash
   sequelize migration:generate --name add_age_to_users
   ```

   This command generates a migration file that you can then edit to define changes to the database schema, such as adding a new column.

4. **Database Synchronization:** The CLI provides commands to synchronize your models with the database, applying any pending migrations. For example:

   ```bash
   sequelize db:migrate
   ```

   This command runs any pending migrations, updating the database schema accordingly.

5. **Seed Generation:** Seed files are used to populate your database with initial data. Sequelize CLI can generate seed files for you:

   ```bash
   sequelize seed:generate --name demo-user
   ```

   This command generates a seed file that you can edit to define the initial data you want to insert into the database.

Sequelize CLI simplifies and automates the process of managing database-related tasks, making it easier for developers to work with Sequelize in a structured and organized way.